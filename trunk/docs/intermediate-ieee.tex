
%% bare_jrnl.tex
%% V1.2
%% 2002/11/18
%% by Michael Shell
%% mshell@ece.gatech.edu
%%
%% NOTE: This text file uses MS Windows line feed conventions. When (human)
%% reading this file on other platforms, you may have to use a text
%% editor that can handle lines terminated by the MS Windows line feed
%% characters (0x0D 0x0A).
%%
%% This is a skeleton file demonstrating the use of IEEEtran.cls
%% (requires IEEEtran.cls version 1.6b or later) with an IEEE journal paper.
%%
%% Support sites:
%% http://www.ieee.org
%% and/or
%% http://www.ctan.org/tex-archive/macros/latex/contrib/supported/IEEEtran/
%%
%% This code is offered as-is - no warranty - user assumes all risk.
%% Free to use, distribute and modify.

% *** Authors should verify (and, if needed, correct) their LaTeX system  ***
% *** with the testflow diagnostic prior to trusting their LaTeX platform ***
% *** with production work. IEEE's font choices can trigger bugs that do  ***
% *** not appear when using other class files.                            ***
% Testflow can be obtained at:
% http://www.ctan.org/tex-archive/macros/latex/contrib/supported/IEEEtran/testflow


% Note that the a4paper option is mainly intended so that authors in
% countries using A4 can easily print to A4 and see how their papers will
% look in print. Authors are encouraged to use U.S. letter paper when 
% submitting to IEEE. Use the testflow package mentioned above to verify
% correct handling of both paper sizes by the author's LaTeX system.
%
% Also note that the "draftcls" or "draftclsnofoot", not "draft", option
% should be used if it is desired that the figures are to be displayed in
% draft mode.
%
% This example can be formatted using the peerreview
% (instead of journal) mode.
\documentclass[journal]{IEEEtran}
% If the IEEEtran.cls has not been installed into the LaTeX system files,
% manually specify the path to it:
% \documentclass[journal]{../sty/IEEEtran}


% some very useful LaTeX packages include:

%\usepackage{cite}      % Written by Donald Arseneau
                        % V1.6 and later of IEEEtran pre-defines the format
                        % of the cite.sty package \cite{} output to follow
                        % that of IEEE. Loading the cite package will
                        % result in citation numbers being automatically
                        % sorted and properly "ranged". i.e.,
                        % [1], [9], [2], [7], [5], [6]
                        % (without using cite.sty)
                        % will become:
                        % [1], [2], [5]--[7], [9] (using cite.sty)
                        % cite.sty's \cite will automatically add leading
                        % space, if needed. Use cite.sty's noadjust option
                        % (cite.sty V3.8 and later) if you want to turn this
                        % off. cite.sty is already installed on most LaTeX
                        % systems. The latest version can be obtained at:
                        % http://www.ctan.org/tex-archive/macros/latex/contrib/supported/cite/

%\usepackage{graphicx}  % Written by David Carlisle and Sebastian Rahtz
                        % Required if you want graphics, photos, etc.
                        % graphicx.sty is already installed on most LaTeX
                        % systems. The latest version and documentation can
                        % be obtained at:
                        % http://www.ctan.org/tex-archive/macros/latex/required/graphics/
                        % Another good source of documentation is "Using
                        % Imported Graphics in LaTeX2e" by Keith Reckdahl
                        % which can be found as esplatex.ps and epslatex.pdf
                        % at: http://www.ctan.org/tex-archive/info/
% NOTE: for dual use with latex and pdflatex, instead load graphicx like:
%\ifx\pdfoutput\undefined
%\usepackage{graphicx}
%\else
%\usepackage[pdftex]{graphicx}
%\fi

% However, be warned that pdflatex will require graphics to be in PDF
% (not EPS) format and will preclude the use of PostScript based LaTeX
% packages such as psfrag.sty and pstricks.sty. IEEE conferences typically
% allow PDF graphics (and hence pdfLaTeX). However, IEEE journals do not
% (yet) allow image formats other than EPS or TIFF. Therefore, authors of
% journal papers should use traditional LaTeX with EPS graphics.
%
% The path(s) to the graphics files can also be declared: e.g.,
% \graphicspath{{../eps/}{../ps/}}
% if the graphics files are not located in the same directory as the
% .tex file. This can be done in each branch of the conditional above
% (after graphicx is loaded) to handle the EPS and PDF cases separately.
% In this way, full path information will not have to be specified in
% each \includegraphics command.
%
% Note that, when switching from latex to pdflatex and vice-versa, the new
% compiler will have to be run twice to clear some warnings.


%\usepackage{psfrag}    % Written by Craig Barratt, Michael C. Grant,
                        % and David Carlisle
                        % This package allows you to substitute LaTeX
                        % commands for text in imported EPS graphic files.
                        % In this way, LaTeX symbols can be placed into
                        % graphics that have been generated by other
                        % applications. You must use latex->dvips->ps2pdf
                        % workflow (not direct pdf output from pdflatex) if
                        % you wish to use this capability because it works
                        % via some PostScript tricks. Alternatively, the
                        % graphics could be processed as separate files via
                        % psfrag and dvips, then converted to PDF for
                        % inclusion in the main file which uses pdflatex.
                        % Docs are in "The PSfrag System" by Michael C. Grant
                        % and David Carlisle. There is also some information 
                        % about using psfrag in "Using Imported Graphics in
                        % LaTeX2e" by Keith Reckdahl which documents the
                        % graphicx package (see above). The psfrag package
                        % and documentation can be obtained at:
                        % http://www.ctan.org/tex-archive/macros/latex/contrib/supported/psfrag/

%\usepackage{subfigure} % Written by Steven Douglas Cochran
                        % This package makes it easy to put subfigures
                        % in your figures. i.e., "figure 1a and 1b"
                        % Docs are in "Using Imported Graphics in LaTeX2e"
                        % by Keith Reckdahl which also documents the graphicx
                        % package (see above). subfigure.sty is already
                        % installed on most LaTeX systems. The latest version
                        % and documentation can be obtained at:
                        % http://www.ctan.org/tex-archive/macros/latex/contrib/supported/subfigure/

%\usepackage{url}       % Written by Donald Arseneau
                        % Provides better support for handling and breaking
                        % URLs. url.sty is already installed on most LaTeX
                        % systems. The latest version can be obtained at:
                        % http://www.ctan.org/tex-archive/macros/latex/contrib/other/misc/
                        % Read the url.sty source comments for usage information.

%\usepackage{stfloats}  % Written by Sigitas Tolusis
                        % Gives LaTeX2e the ability to do double column
                        % floats at the bottom of the page as well as the top.
                        % (e.g., "\begin{figure*}[!b]" is not normally
                        % possible in LaTeX2e). This is an invasive package
                        % which rewrites many portions of the LaTeX2e output
                        % routines. It may not work with other packages that
                        % modify the LaTeX2e output routine and/or with other
                        % versions of LaTeX. The latest version and
                        % documentation can be obtained at:
                        % http://www.ctan.org/tex-archive/macros/latex/contrib/supported/sttools/
                        % Documentation is contained in the stfloats.sty
                        % comments as well as in the presfull.pdf file.
                        % Do not use the stfloats baselinefloat ability as
                        % IEEE does not allow \baselineskip to stretch.
                        % Authors submitting work to the IEEE should note
                        % that IEEE rarely uses double column equations and
                        % that authors should try to avoid such use.
                        % Do not be tempted to use the cuted.sty or
                        % midfloat.sty package (by the same author) as IEEE
                        % does not format its papers in such ways.

%\usepackage{amsmath}   % From the American Mathematical Society
                        % A popular package that provides many helpful commands
                        % for dealing with mathematics. Note that the AMSmath
                        % package sets \interdisplaylinepenalty to 10000 thus
                        % preventing page breaks from occurring within multiline
                        % equations. Use:
%\interdisplaylinepenalty=2500
                        % after loading amsmath to restore such page breaks
                        % as IEEEtran.cls normally does. amsmath.sty is already
                        % installed on most LaTeX systems. The latest version
                        % and documentation can be obtained at:
                        % http://www.ctan.org/tex-archive/macros/latex/required/amslatex/math/



% Other popular packages for formatting tables and equations include:

%\usepackage{array}
% Frank Mittelbach's and David Carlisle's array.sty which improves the
% LaTeX2e array and tabular environments to provide better appearances and
% additional user controls. array.sty is already installed on most systems.
% The latest version and documentation can be obtained at:
% http://www.ctan.org/tex-archive/macros/latex/required/tools/

% Mark Wooding's extremely powerful MDW tools, especially mdwmath.sty and
% mdwtab.sty which are used to format equations and tables, respectively.
% The MDWtools set is already installed on most LaTeX systems. The lastest
% version and documentation is available at:
% http://www.ctan.org/tex-archive/macros/latex/contrib/supported/mdwtools/


% V1.6 of IEEEtran contains the IEEEeqnarray family of commands that can
% be used to generate multiline equations as well as matrices, tables, etc.


% Also of notable interest:

% Scott Pakin's eqparbox package for creating (automatically sized) equal
% width boxes. Available:
% http://www.ctan.org/tex-archive/macros/latex/contrib/supported/eqparbox/



% Notes on hyperref:
% IEEEtran.cls attempts to be compliant with the hyperref package, written
% by Heiko Oberdiek and Sebastian Rahtz, which provides hyperlinks within
% a document as well as an index for PDF files (produced via pdflatex).
% However, it is a tad difficult to properly interface LaTeX classes and
% packages with this (necessarily) complex and invasive package. It is
% recommended that hyperref not be used for work that is to be submitted
% to the IEEE. Users who wish to use hyperref *must* ensure that their
% hyperref version is 6.72u or later *and* IEEEtran.cls is version 1.6b
% or later. The latest version of hyperref can be obtained at:
%
% http://www.ctan.org/tex-archive/macros/latex/contrib/supported/hyperref/
%
% Also, be aware that cite.sty (as of version 3.9, 11/2001) and hyperref.sty
% (as of version 6.72t, 2002/07/25) do not work optimally together.
% To mediate the differences between these two packages, IEEEtran.cls, as
% of v1.6b, predefines a command that fools hyperref into thinking that
% the natbib package is being used - causing it not to modify the existing
% citation commands, and allowing cite.sty to operate as normal. However,
% as a result, citation numbers will not be hyperlinked. Another side effect
% of this approach is that the natbib.sty package will not properly load
% under IEEEtran.cls. However, current versions of natbib are not capable
% of compressing and sorting citation numbers in IEEE's style - so this
% should not be an issue. If, for some strange reason, the user wants to
% load natbib.sty under IEEEtran.cls, the following code must be placed
% before natbib.sty can be loaded:
%
% \makeatletter
% \let\NAT@parse\undefined
% \makeatother
%
% Hyperref should be loaded differently depending on whether pdflatex
% or traditional latex is being used:
%
%\ifx\pdfoutput\undefined
%\usepackage[hypertex]{hyperref}
%\else
%\usepackage[pdftex,hypertexnames=false]{hyperref}
%\fi
%
% Pdflatex produces superior hyperref results and is the recommended
% compiler for such use.



% *** Do not adjust lengths that control margins, column widths, etc. ***
% *** Do not use packages that alter fonts (such as pslatex).         ***
% There should be no need to do such things with IEEEtran.cls V1.6 and later.


% correct bad hyphenation here
\hyphenation{op-tical net-works semi-conduc-tor}


\begin{document}
%
% paper title
\title{Grayscale image segmentation using reversible jump MCMC.}
%
%
% author names and IEEE memberships
% note positions of commas and nonbreaking spaces ( ~ ) LaTeX will not break
% a structure at a ~ so this keeps an author's name from being broken across
% two lines.
% use \thanks{} to gain access to the first footnote area
% a separate \thanks must be used for each paragraph as LaTeX2e's \thanks
% was not built to handle multiple paragraphs
\author{Pavel V. Senin,
			  ~\IEEEmembership{UH,~Student}% <-this % stops a space
\thanks{The work was not supported by the IEEE.}% <-this % stops a space
\thanks{Last revised at \date{\today}.}% <-this % stops a space
\thanks{Thanks to my kids for constant disturbing.}}
% note the % following the last \IEEEmembership and also the first \thanks - 
% these prevent an unwanted space from occurring between the last author name
% and the end of the author line. i.e., if you had this:
% 
% \author{....lastname \thanks{...} \thanks{...} }
%                     ^------------^------------^----Do not want these spaces!
%
% a space would be appended to the last name and could cause every name on that
% line to be shifted left slightly. This is one of those "LaTeX things". For
% instance, "A\textbf{} \textbf{}B" will typeset as "A B" not "AB". If you want
% "AB" then you have to do: "A\textbf{}\textbf{}B"
% \thanks is no different in this regard, so shield the last } of each \thanks
% that ends a line with a % and do not let a space in before the next \thanks.
% Spaces after \IEEEmembership other than the last one are OK (and needed) as
% you are supposed to have spaces between the names. For what it is worth,
% this is a minor point as most people would not even notice if the said evil
% space somehow managed to creep in.
%
% The paper headers
\markboth{Final report, ~ICS663 ~non-classified ~files ~FALL 2006, ~Vol.~1, \date{\today}}{Shell \MakeLowercase{\textit{et al.}}: Very first draft.}
% The only time the second header will appear is for the odd numbered pages
% after the title page when using the twoside option.
% 
% *** Note that you probably will NOT want to include the author's name in ***
% *** the headers of peer review papers.                                   ***

% If you want to put a publisher's ID mark on the page
% (can leave text blank if you just want to see how the
% text height on the first page will be reduced by IEEE)
%\pubid{0000--0000/00\$00.00~\copyright~2002 IEEE}

% use only for invited papers
%\specialpapernotice{(Invited Paper)}

% make the title area
\maketitle


\begin{abstract}
While the ultimate goal of this project was to build a Reversible Jump MCMC sampler for image segmentation the others methods such as ICM, Gibbs sampler, Metropolis sampler and Simulated Annealing arised as a necessary milestones in order to meet the goal. So far all milestones are implemented while RJMCMC sampler itself is not done - the acceptance probability of the split or joining of classes is not finished. Being crucial in RJMCMC method, this part prevents from successful project closure. Nevertheless author believe that it could and would be done in short time after 5th of December of 2006. If not, the way and idea are well expressed in cited articles, this report and provided to the public source code at http://code.google.com/p/rjimage/. This is SVN repository URL that contains this documents along with software implementation.
\end{abstract}

\begin{keywords}
RJMCMC, image segmentation, Gibbs sampler, ICM, Metropolis sampler.
\end{keywords}
% Note that keywords are not normally used for peerreview papers.

% For peer review papers, you can put extra information on the cover
% page as needed:
% \begin{center} \bfseries EDICS Category: 3-BBND \end{center}
%
% For peerreview papers, inserts a page break and creates the second title.
% Will be ignored for other modes.
\IEEEpeerreviewmaketitle


\section{Model design}
\subsection{Introduction}
\PARstart{I}{n} the core of this work lies my own interest for the RJMCMC and publication [1] by Zoltan Kato "Bayesian Color image segmentation using reversible jump Markov Chain Monte-Carlo". I have chosed easiest way to go - grayscale, but as could be seen it doesn't really matter, the results could be easily elevated to color images encoded in any chosen color space. RJMCMC method described by Green [5] and used by Kato [1] deals with the problem of unsupervised image segmentation. It means that it should accept an image as the only input parameter and produce a segmented image representation as output. The method itself relies on the use of a firs-order Markov Field (MRF, Plotts model) and employs ICM algorithm [2], Gibbs sampler [3], and Metropolis sampler [4] along with optimization criteria based on Simulated annealing algorithm.

As I already written in the project proposal and intermediate report, the image segmentation is one of the fundamental problems in computer vision and lies an the first-row low-level tasks for full line of image processing like surface description, object recognition, content bases indexing etc. 

I believe that Reversible Jump mimics natural human ability to segment images that are noisy or have very little features to do segmentation exactly. I think that humans also just trying some segmentation and choose the most probable one using their experience. Basically RJMCMC method search within the multidimensional variants space trying to maximize classes separation. The approach itself seems to be heavyweight and computationally costly, but second reason to choose this as I mentioned is my own interest in understanding of unsupervised learning and ICA. 

At the beginning I was considering R (or 'GNU S'), a freely available language and environment for statistical computing and graphics as the programming environment. The taken choice was erroneous, despite being well supported and having numerous libraries for image IO and raster manipulations along with unlimited statistical tools, the R is extremely slow in matrix processing and has certain limitations in software design due to common namespace, plain code execution and lack of an IDE and debugger. The lesson learned from this failure is that while R is the ultimate "statistic" calculator (due to numerous packages) along with great plotting capabilities, it is more appropriate for single task solving, or conceptual coding, not for the routine program development, debug and evolution of software.

My second choice was Java and so far it provides desired functionality and flexibility. The software that done for the project is perfectly working, has nice GUI and could be easily debugged and extended in any desired way.

\subsection{Model design}
According [1], the observed image is: $\textsl{F} = \left\{ \vec{f}_{s} \left| s \in \textsl{S}, \forall i : 0 < \vec{f}^{i}_{s} < 1 \right. \right\}$,
where vector $\vec{f}_{s}$ is vector that carried intensity of color for pixel s. The segmentation itself is just labeling of each pixel $s \in S$ by label $\omega_{s} \in \Lambda = \left\{ 1,2,...,L \right\}$. $\omega\in\Omega$ denotes a labeling (or segmentation), $\Omega$ is a set of all possible labeling.

We regard our image as a sample drawn from unknown Gaussian mixture distribution. The goal of our analysis is inference about the number 
$L$ components:

- the component parameter 

\begin{flushright}
$\Theta = \left\{\forall\lambda : 1\leq \lambda \leq \textsl{L}, \Theta_{\lambda}=\left(\vec{\mu_{\lambda}},\Sigma_{\lambda} \right) \right\};$
\end{flushright}

- the component weights $\textsl p_{\lambda}\left(1\leq\lambda\leq \textsl L \right)$ summing to 1;

- the clique potential (or inter-pixel interaction strength), $\beta$;
 
- the segmentation $\omega$.

The joint distribution of above variables $L,p,\beta,\omega,\Theta,F$ given by formula :
\begin{equation}
P\left(L,p,\beta,\omega,\Theta,F\right) = P\left(\omega,F\left|\Theta,\beta,p,L\right.\right)P\left(\Theta,\beta,p,L\right)
\end{equation}
Impose the independence of labeling $\Theta$, inter-pixel composition $\beta$, component weights $p$, and labels set power $L$ as parameters that given randomly in every processed image. Therefore their joint probability reduces to 
\begin{equation}
P\left(\Theta,\beta,p,L\right) = P\left(\Theta\right)P\left(\beta\right)P\left(p\right)P\left(L\right)
\end{equation}
The posterior distribution of $\left(F, \omega \right)$ may be expressed as:
\begin{equation}
P\left(F,\omega\left|\right.\Theta,\beta,p,L\right)=P\left(F\left|\right.\omega,\Theta,\beta,p,L\right)P\left(\omega\left|\right.\Theta,\beta,p,L\right)
\end{equation}
The pixel classes (segmentation itself) represented as a multivariate Gaussian distribution and the underlied MRF (Markov Random Field) process follows a Gibbs distribution defined over a first order neighborhood system. 

Previous equation could be factored out as:
\begin{eqnarray}
\begin{group}
\left(F\left|\right.\omega,\Theta,\beta,p,L\right) = 
P\left(F\left|\right.\omega,\Theta\right)=\nonumber \\
\begin{flushleft}
\prod_{s \in S}
\left(\frac{1}{\sqrt{\left(2\pi\right)^{3}\left|\sum_{\omega_{s}}\right|}}
\exp\left(
-\frac{1}{2}
\left(\vec{f}_{s}-\vec{\mu}_{\omega_{s}}\right) \right.
\end{flushleft}\times\\
\begin{flushright}
\left.\left.\sum_{\omega_{s}}^{-1}
\left(\vec{f}_{s}-\vec{\mu}_{\omega_{s}}\right)^T \right)\right)
\end{flushright}
\end{group}
\end{eqnarray}


Proceeding further we have:
\begin{equation}
P\left(\omega\left|\right.\Theta,\beta,p,L\right) = 
P\left(\omega\left|\right.\beta,p,L\right) = $$
\par
$\frac {1} {Z\left(\beta,p,L\right)}
\exp\left(
-U\left(\omega\left|\right.\beta p,L\right)
\right)
\end{equation}

where $U\left(\omega\left|\right.\beta, p, L \right)$ is energy function:

\begin{equation}
U\left(\omega\left|\right.\beta, p,L\right) = 
\sum_{s\in S}-\log\left(p_{\omega_{s}}\right) +
\beta\sum_{\left\{s,r\right\}\in C}
\delta\left(\omega_{s},\omega_{r}\right)
\end{equation}
$\delta\left(\omega_{s},\omega_{r}\right) = 1$ if $\omega_{s}$ and $\omega_{r}$ are different and -1 otherwise. 
$Z\left(\beta,p,L\right) = \sum_{\omega\in \Omega}\exp\left( -U\left(\omega\left|\right.\beta,p,L\right)\right)$ is normalization constant or partition function. $C$ denotes the set of cliques and $\left\{s,r\right\}$ is doubleton containing the neighboring pixel sites $r$ and $s$.

Note that the whole posterior distribution could be derived from Gibbs distribution where the Gaussian distribution taken in account in the energy of the external field:
\begin{equation}
U\left(F\left|\right.\omega,\Theta\right) = 
-\log\left(P\left(F\left|\right.\omega,\Theta\right)\right) = 

\sum\limits_{s\in S}\left(
\ln\left(\sqrt{\left(2\pi\right)^{3}}\right) +
\frac{1}{2}
\left(\vec{f}_{s}-\vec{\mu}_{\omega_{s}}\right)
\sum_{\omega_{s}}^{-1}
\left(\vec{f}_{s}-\vec{\mu}_{\omega_{s}}\right)^T
\right)
\end{equation}

Now, we are using facts that $P\left(F\right)$ is constant for any particular image and Equation (1), Equation (2), Equation (4) and Equation (7) we are able to approximate the posterior density $P\left(L,p,\beta,\omega,\Theta\left|\right.F\right) = P\left(L,p,\beta,\omega,\Theta,F\right)/P\left(F\right)$:
\begin{equation}
P\left(L,p,\beta,\omega,\Theta\left|\right.F\right) = $$
\par
$P\left(F\left|\right.\omega\Theta\right)
P\left(\omega\left|\right.\beta,p,L\right)
P\left(\Theta\right)P\left(\beta\right)P\left(p\right)P\left(L\right)$$
\par
$\approx
\prod\limits_{s \in S}
\left(
\frac{1}
{\sqrt{\left(2\pi\right)^{3}\left|\sum_{\omega_{s}}\right|}}
\exp\left(
-\frac{1}{2}
\left(\vec{f}_{s}-\vec{\mu}_{\omega_{s}}\right)\right.\right. \times $$
\par
$\left.\left.\sum_{\omega_{s}}^{-1}
\left(\vec{f}_{s}-\vec{\mu}_{\omega_{s}}\right)^T\right)\right) \times $$
\par
$\prod\limits_{s\in S}
\frac{
p_{\omega_{s}}
\exp\left(
-\beta 
\sum_{\forall r : \left\{s,r\right\} \in C } \delta\left(\omega_{s},\omega_{r}\right)
\right)
}
{
\sum_{\lambda\in\Lambda}p_{\lambda}
\exp\left(
-\beta
\sum_{\forall r : \left\{s,r\right\} \in C } \delta\left(\lambda,\omega_{r}\right)
\right)
}

\times P\left(\beta\right)P\left(L\right)
\prod\limits_{\lambda\in\Lambda}
P\left(\vec{\mu_{\lambda}}\right)P\left(\Sigma_{\lambda}\right)P\left(p_{\lambda}\right)
\end{equation}

Concerning the priors, the published model follows [6] and [7] by choosing uniform reference for priors for $L$, $\vec{\mu_{\lambda}}$, $\Sigma_{\lambda}$, $p_{\lambda}$ where $\lambda \in \Lambda$.

\subsection{Sampling from Posterior}
Having Equation (8) as the base point the original article proceed to the MCMC algorithm that is used to sample from the whole posterior distribution in order to obtain a MAP estimate via simulated annealing [8]. However, classical MCMC methods are restricted to problems where the dimensionality of the parameter vector is fixed. Therefore, the estimation of the number of mixture components is not possible. In order to overcome this obstacle model uses Reversible Jump MCMC (RJMCMC), that has been proposed by Peter Green in [5]. This method makes it possible to construct reversible Markov chain samplers that jump between parameter subspaces of different dimensionality. RJMCMC allows the direct sampling of the whole posterior distribution defined over the combined model space thus reducing the optimization process to a single simulated annealing run. Another advantage is that no coarse segmentation neither exhaustive search over a parameter subspace is required. Our set of unknowns is $\left\{L,p,\beta,\omega,\Theta\right\}$, lets denote it as $X$ and let $\pi\left(X\right)$ be the target probability measure (the posterior distribution). The wildly used tool to sample from such distribution is Metropolis-Hasting method [3, 4]. When the current state is $X$ the new state is drawn from an arbitrary joint distribution $q\left(X,X'\right)$. The new state is accepted with probability 
\begin{equation}
A(X,X') = \min\left(1,
\frac{\pi\left(X'\right)q\left(X,X'\right)}
{\pi\left(X\right)q\left(X',X\right)}\right)
\end{equation}
The presented model has multiple parameter subspaces of different dimensionality and it is necessary to devise move types between subspaces authors made decision to build the hybrid sampler in the five moves:
\begin{enumerate}
	\item Sampling the class labels $\omega$, i.e. resegment the image;
	\item Sampling Gaussian parameters $\Theta=\left\{\left(\vec_{\mu_{\lambda},\Sigma_{\lambda}\right)\left|\right.\lambda\in\Lambda\right\}$;
	\item Sampling the mixture weights $p_{\lambda}\left(\lambda\in\Lambda\right)$;
	\item Sampling the MRF hyperparameter $\beta$;
	\item Sampling the number of classes $L$, i.e. splitting one of mixture components into two or merge two of them into one.
\end{enumerate}
The only randomness left in there is the choice between splitting and merging in move (5). One iteration of the hybrid sampler, also
called a sweep by authors, consists in a complete pass over these moves. The first four move types are conventional in the sense that they do not alter the dimension of the parameter space. Concerning the fifth move type, the reversible jump mechanism is needed. 

\section{Hybrid sampler}
From here I will point out the moves that my implementation doing and problem that prevents task from completion.

\subsection{Move 1 - image segmentation}
This move is just classical image segmentation with known parameters, i.e. we have fixed parameters - their estimates and segmentation reduces to: 
\clearpage
\begin{equation}
\begin{flushleft}
P\left(L,p,\beta,\omega,\Theta\left|\right.F\right) \approx $$
\end{flushleft}
\par
$P\left(F\left|\right.\omega,\widehat{\Theta}\right)
P\left(\omega\left|\right.\widehat{\beta},\widehat{p},\widehat{L}\right)
P\left(\Theta\right)P\left(\beta\right)P\left(p\right)P\left(L\right)\approx$$
\par
\begin{flushleft}
$
\prod\limits_{s \in S}
\left(
\frac{1}
{\sqrt{\left(2\pi\right)^{3}\left|\widehat{\Sigma}_{\omega_{s}}\right|}}
\exp\left(
-\frac{1}{2}
\left(\vec{f}_{s}-\vec{\widehat{\mu}}_{\omega_{s}}\right)\widehat{\Sigma}_{\omega_{s}}^{-1}
\left(\vec{f}_{s}-\vec{\widehat{\mu}}_{\omega_{s}}\right)^T
\right)
\right)
\end{flushright}

\times
\prod\limits_{s\in S}
\widehat{p}_{\omega_{s}}
\exp\left(
-\widehat{\beta} 
\sum\limits_{\forall r : \left\{s,r\right\} \in C } \delta\left(\omega_{s},\omega_{r}\right)
\right)
\end{equation}
The author points here that this sub-chain of classical segmentation parameters should be obtained by Gibbs sampler [9] since $\omega$ takes discrete values over finite set $\Lambda$. In order to proceed further I have implemented Gibbs sampler that does pixel labeling once given the set of Gaussians. This part could be run independently in my demo software. The results of Gibbs sampler optimized using Simulated Annealing with a logarithmic cooling schedule that chosen so that the algorithm would converge after a reasonable number of iterations. The schedule is given by:
$T_{k+1} = 0.98T_{k}$ with an initial temperature set to 20.0.

\subsection{Move 2 - estimating Gaussian parameters}
This move is aiming mean and covariance matrix of the pixel classes. By setting variables $L,p,\beta,\omega$ to their estimates $\widehat{L},\widehat{p},\widehat{\beta},\widehat{\omega}$ Equation (8) reduces to the form:
\begin{equation}
P\left(L,p,\beta,\omega,\Theta\left|\right.F\right) \approx 
P\left(F,\widehat{\omega}\left|\right.\Theta\right)P\left(\Theta\right) = 

\prod\limits_{\lambda \in \Lambda}{
\prod\limits_{s:\widehat{\omega}_{s}=\lambda}{
P\left(\vec{f}_{s}\right|\left.\vec{\mu}_{\lambda},\Sigma_{lambda}\right)
P\left(\vec{\mu}_{\lambda}\right)P\left(\Sigma_{\lambda}\right) = 

\times
\prod\limits_{s\in S}
\frac{1}
{\left(\left(2\pi\right)^{3}
\left|\Sigma_{\lambda}\right|\right)^{\left|\Sigma_{\lambda}\right|/2}
}
\exp{
\left(
-\frac{1}{2}
\sum\limits_{s:\widehat{\omega}_{s}=\lambda}\left(\vec{f}_{s}-\vec{\mu}_{\lambda}\right)\right.\right.$$
\par
$\left.\left.\Sigma_{\lambda}^{-1}\left(\vec{f}_{s}-\vec{\mu}_{\lambda}\right)^{T}
\right)
}
\times
\prod\limits_{\lambda\in\Lambda}{
P\left(\vec{\mu}_{\lambda}\right)P\left(\Sigma_{\lambda}\right)P\left(p_{\lambda}\right)
}
\end{equation}


where $\left|\Sigma_{\lambda}\right|$ is number of sites labeled by $\lambda$.

\subsection{Move 3 - sampling Mixture Weights}
This move type aims at estimating the mixture weights $p$. These weights are incorporated into the Gibbs distribution of the underlying label process $\omeag$ as the external field strength. In this model the weights required to be normalized by imposing the following constraint:
\begin{equation}
\sum_{\lambda\in\Lambda}p_{\lambda}=1
\end{equation} 
having this constraint it is possible to fix hyperparameter $\beta$ a priori and authors fixed it to 2.5. Using these conditions and setting variables $L,\beta,\omega,\Theta$ to their estimates $\widehat{L},\widehat{\beta},\widehat{\omega},\widehat{\Theta}$
\begin{equation}
P\left(L,p,\beta,\omega,\Theta\left|\right.F\right) \approx 
P\left(\widehat{\omega}\left|\right.\widehat{\beta},\widehat{p},\widehat{L}\right)P\left(p\right) =$$
\par
$ \prod\limits_{\lambda \in \Lambda}P\left(p_{\lambda}\right)
\left(\frac{p_{\lambda}}{\sum\limits_{\lambda\in\Lambda}{p_{\lambda}}}\right)^{\left|S_{\lambda}\right|} \times $$
\par
$\prod\limits_{s\in S}
\frac{
\exp\left(
-\widehat{\beta}
\sum\limits_{\forall r : \left\{s,r\right\} \in C } \delta\left(\widehat{\omega}_{s},\widehat{\omega}_{r}\right)
\right)\sum\limits_{\lambda\in\Lambda}p_{\lambda}
}
{\sum\limits_{\lambda\in\Lambda}p_{\lambda}
\exp\left(
-\widehat{\beta}
\sum\limits_{\forall r : \left\{s,r\right\} \in C } \delta\left(\lambda,\widehat{\omega}_{r}\right)
\right)} = $$
\par
$\prod\limits_{\lambda\in \Lambda}
P\left(p_{\lambda}\right)p_{\lambda}^{\left|S_{\lambda}\right|} \times
\prod\limits_{s \in S}
\frac{
\exp\left(
-\widehat{\beta}
\sum\limits_{\forall r : \left\{s,r\right\} \in C } \delta\left(\widehat{\omega}_{s},\widehat{\omega}_{r}\right)
\right)
}
{\sum\limits_{\lambda\in\Lambda}p_{\lambda}
\exp\left(
-\widehat{\beta}
\sum\limits_{\forall r : \left\{s,r\right\} \in C } \delta\left(\lambda,\widehat{\omega}_{r}\right)
\right)}
\end{equation} 

\subsection{Move 4 - sampling Hyperparameter \beta}
As already mentioned this model has $\beta$ parameter fixed a priory. The value chosen is 2.5.

\section{Estimating Number of classes}
Herein, the cornerstone of the method goes. This move type involves changing $L$ by $1$ and making necessary corresponding changes to $\omega,\Theta $ and $p$. Should be noted that right before this move all posterior distribution from equation (8) is already sampled, $\beta$ is set to its value and we have following distribution (from Equation (8)):
\begin{equation}
P\left(L,p,\beta,\omega,\Theta\left|\right.F\right) = 
$$
\par
$
P\left(F\left|\right.\omega\Theta\right)
P\left(\omega\left|\right.\widehat{\beta},p,L\right)
P\left(\Theta\right)P\left(\beta\right)P\left(p\right)P\left(L\right)\approx
$$
\par
$
\prod\limits_{s \in S}
\left(
\frac{1}
{\sqrt{\left(2\pi\right)^{3}\left|\sum\limits_{\omega_{s}}\right|}}
\exp\left(
-\frac{1}{2}
\left(\vec{f}_{s}-\vec{\mu}_{\omega_{s}}\right)\right.\right. \times 
$$
\par
$
\left.\left.\sum_{\omega_{s}}^{-1}
\left(\vec{f}_{s}-\vec{\mu}_{\omega_{s}}\right)^T\right)\right) =
$$
\par
$
\prod\limits_{s\in S}
\frac{
p_{\omega_{s}}
\exp\left(
-\widehat{\beta}
\sum\limits_{\forall r : \left\{s,r\right\} \in C } \delta\left(\omega_{s},\omega_{r}\right)
\right)
}
{
\sum\limits_{\lambda\in\Lambda}p_{\lambda}
\exp\left(
-\widehat{\beta}
\sum\limits_{\forall r : \left\{s,r\right\} \in C } \delta\left(\lambda,\omega_{r}\right)
\right)
}\times 
P\left(L\right)
\prod\limits_{\lambda\in\Lambda}
P\left(\vec{\mu_{\lambda}}\right)P\left(\Sigma_{\lambda}\right)P\left(p_{\lambda}\right)
\end{equation}

Since the dimensionality of the parameter space is altered, the reversible jump technique, is needed for sampling from the posterior distribution. I will not go into the deepness of mathematical formulas in this report due to space and IEEE format constraints, all of those could be found within original article [1]. 

But I would try to give the idea of RJMCMC using plain English language. The rule that should be obeyed in the reversible jump is reversibility, i.e. if the system could move from one dimensionality to another it must have the same probability to move back. Underlying rule of equality of splitting/joining probabilities allow the system to do exhaustive search over all possible combinations (segmentations in this case). Therefore probabilities of split and join are calculated at first, than acceptance probability of split and joining compared and final decision made.

\emph{Splitting the class}. The split proposal begins by choosing a class $\lambda$ $at~random$ with a uniform probability $P_{select}^{split}\left(\lambda\right) = 1/L$, Then L is increased by 1 and $\lambda$ is splitted into $\lambda_{1}$ and $\lambda_{2}$. In doing so, a new set of parameters need to be generated. The generation is driven by random variables that are chosen from the interval (0; 1]. In order to favor splitting the class into roughly equal portions, beta distributions ( $beta(1.1,1.1)$) is used.

After splitting the problem of reallocation of those sites $s\inS$ where $\widehat{\omega_{s}}=\lambda$ arises. A tentative labeling of the sites could be sampled using Gibbs sampler. However, a labeling which has a relatively high posterior probability needed to meet acceptance probability. To achieve this goal, author uses ICM [2] algorithm to obtain a suboptimal segmentation. The obtained $\omega^{+}$ has a relatively high posterior probability since the ICM-ed tentative labeling is close to the optimal labeling. To follow the author in this move I have implemented ICM algorithm that uses The same Simulated Annealing with the same logarithmic schedule. In my demo software it could be run by button "ICM".

\emph{Merging two classes}.
To decide which pair of classes has to be merged all pairwise distances $d\left(\lambda_{1},\lambda_{2}\right)$ computed using Mahalanobis distance and probability of selecting pair for merging is relative to their distance:
\begin{equation}
P_{select}^{merge}\left(\lambda_{1},\lambda_{2}\right) = \frac{d\left(\lambda_{1},\lambda_{2}\right)}{\sum\limits_{\labda\in\Lambda}\sum\limits_{k\in\Lambda}d\left(\lambda,k\right)}
\end{equation}
The merge proposal is deterministic, once the choice of $\lambda_{1}$ and $\lambda_{2}$ has been made. The two components are merged reducing $L$ by 1.
The reallocation is simply done by setting the label at sites $s\in S_{\lambda_{1},\lambda_{2}}$ \emph{In order to calculate acceptance probability author says that 5 equations should be solved to get random variables that used for its calculation. Here I am facing hard time - I cannot solve those so far . I am still working figuring it out which way it could be done}.

\emph{Acceptance probability}. Once proposal for the split or merging are done acceptance probabilities are computed and compared in order to do the image segmentation.
\begin{equation}
A_{split}\left(L,\widehat{p},\widehat{\beta},\widehat{\omega},\widehat{\Theta};L+1,p^{+},\widehat{\beta},\widehat{\omega}^{+},\widehat{\Theta}^{+}\right) = min\left(1,A\right)
\end{equation}
where 
\begin{equation}
A=\frac{P\left(L+1,p^{+},\widehat{\beta},\widehat{\omega}^{+},\widehat{\Theta}^{+}\left|\right.F\right)}
{P\left(L,\widehat{p},\widehat{\beta},\widehat{\omega},\widehat{\Theta}\left|\right.F\right)}
\times$$
\par
$\frac{P_{merge}\left(L+1\right)P_{select}^{merge}\left(\lambda_{1},\lambda_{2}\right)}
{P_{split}\left(L\right)P_{select}^{split}\left(\lambda\right)P_{realloc}}
\times$$
\par
$\frac{1}
{P_{beta\left(1.1,1.1\right)}\left(u_{1}\right)P_{beta\left(1.1,1.1\right)}\left(u_{2}\right)P_{beta\left(1.1,1.1\right)}\left(u_{3}\right)
}\times$$
\par
$\left|\frac{
\partial\varphi
}{
\partial\left(\Theta_{\lambda},p_{\lambda},u\right)
}\right|
\end{equation}
$P_{realloc}$ denotes the probability of reallocating pixels labeled by $\lambda$ into regions labeled by $\lambda_{1}$ and $\lambda_{2}$. The last factor is Jacobian of the transformation. The acceptance probability now could be expressed as:
\begin{equation}
A_{merge}\left(L,\widehat{p},\widehat{\beta},\widehat{\omega},\widehat{\Theta};L-1,p^{-},\widehat{\beta},\widehat{\omega}^{-},\widehat{\Theta}^{-}\right) = min\left(1,\frac{1}{A}\right)
\end{equation}


\section{Optimization according to the MAP criteria, Simulated annealing}
In the article MAP estimator build using Simulating annealing and provides us with an image segmentation $\widehat{\omega}$ and model parameters $\widehat{L},\widehat{p},\widehat{\beta},\widehat{\Theta}$. The MAP estimator of unknowns is given by combinatorial optimization problem:d using Simulating annealing and provides us with an image segmentation $\widehat{\omega}$ and model parameters $\widehat{L},\widehat{p},\widehat{\beta},\widehat{\Theta}$. The MAP estimator of unknowns is given by combinatorial optimization problem:
\begin{equation}
\left(\widehat{\omega},\widehat{L},\widehat{p},\widehat{\beta},\widehat{\Theta}\right)^{\left(MAP\right)} = 
\arg \max\limits_{L,p,\beta,\omega,\Theta}P\left(L,p,\beta,\omega,\Theta\left|\right.F\right) 
\end{equation}
with the following constraints:
\begin{equation}
\omega \in \Omega,
\end{equation}
\begin{equation}
L_{min} \leq L \leq L_{max},
\end{equation}
\begin{equation}
\sum\limits_{\lambda\in\Lambda}p_{lambda} = 1,
\end{equation}
\begin{equation}
\forall\labda\in\Lambda : 0 \leq \mu_{\labda,i} \leq 1,
\end{equation}
\begin{equation}
\forall\labda\in\Lambda : 0 \leq \Sigma_{\labda,i,i} \leq 1, -1 \leq \Sigma_{\labda,i,j} \leq 1
\end{equation}


\section{Algorithm, RJMCMC segmentation}
In this section I will describe in short algorithm that summarizes all the above:

\begin{enumerate}
	\item Set $k=0$ and initialize $\widehat{L}^{0},\widehat{p}^{0},\widehat{\beta}^{0},\widehat{\Theta}^{0}$, and set temperature $\tau^{0}$.
	\item A sample $\left(\widehat{\omega}^{k},\widehat{L}^{k},\widehat{p}^{k},\widehat{\beta}^{k},\widehat{\Theta}^{k}\right)$ is drawn from the modified posterior distribution using the hybrid sampler defined in section (NUMBER NEEDED) before. The modification is due to simulated annealing constraint - temperature $\tau_{k}$:
\begin{equation}
\prod\limits_{s\in S}
\frac{1}
{\left(\left(2\pi\right)^{3}
\left|\Sigma_{\omega_{s}}\right|\right)^{1/2\tau_{k}}
}
\exp{
\left(
-\frac{1}{2\tau_{k}}
\left(\vec{f}_{s}-\vec{\mu}_{\omega_{s}}\right)
\Sigma_{\omega_{s}}^{-1}\left(\vec{f}_{s}-\vec{\mu}_{\omega_{s}}\right)^{T}
\right)

\times
\prod\limits_{s\in S}\frac
{\exp\left(\frac{\log\left(p_{\omega_{s}}\right)}{\tau_{k}}
-\frac{\beta}{\tau_{k}}\sum\limits_{\forall r : \left\{s,r\right\}\in C}\delta\left(\omega_{s},\omega_{r}\right)\right)}
{
\sum\limits_{\lambda\in\Lambda}
\exp\left(\frac{\log\left(p_{\lambda}\right)}{\tau_{k}}
-\frac{\beta}{\tau_{k}}\sum\limits_{\forall r : \left\{s,r\right\}\in C}\delta\left(\lambda,\omega_{r}\right)\right)}
} 
 \end{equation}
 \begin{enumerate} 
	\item $\widehat{\omega}^{k}$ is drawn from distribution in Equation (10).
   \item $\widehat{\Theta}^{k}$ is drawn from distribution in Equation (11).
   \item $\widehat{p}^{k}$ is drawn from distribution in Equation (13).
   \item $\widehat{L}^{k}$ is estimated using the reversible jump techniques from the section (III).
 \end{enumerate}
 \item GOTO to step 2 with $k=k+1$ and $T_{k+1}$ until $k\leq K$.
\end{enumerate}





% The very first letter is a 2 line initial drop letter followed
% by the rest of the first word in caps.
% 
% form to use if the first word consists of a single letter:
% \PARstart{A}{demo} file is ....
% 
% form to use if you need the single drop letter followed by
% normal text (unknown if ever used by IEEE):
% \PARstart{A}{}demo file is ....
% 
% Some journals put the first two words in caps:
% \PARstart{T}{his demo} file is ....
% 
% Here we have the typical use of a "T" for an initial drop letter
% and "HIS" in caps to complete the first word.
%\PARstart{T}{his} demo file is intended to serve as a ``starter file"
%for IEEE journal papers produced under \LaTeX\ using IEEEtran.cls version
%1.6b and later.
% You must have at least 2 lines in the paragraph with the drop letter
% (should never be an issue)
% May all your publication endeavors be successful.

\hfill
 
\hfill \date{\today}

%\subsection{Subsection Heading Here}
%Subsection text here.

% needed in second column of first page if using \pubid
%\pubidadjcol

%\subsubsection{Subsubsection Heading Here}
%Subsubsection text here.

% Reminder: the "draftcls" or "draftclsnofoot", not "draft", class option
% should be used if it is desired that the figures are to be displayed while
% in draft mode.

% An example of a floating figure using the graphicx package.
% Note that \label must occur AFTER (or within) \caption.
% For figures, \caption should occur after the \includegraphics.
%
%\begin{figure}
%\centering
%\includegraphics[width=2.5in]{myfigure}
% where an .eps filename suffix will be assumed under latex, 
% and a .pdf suffix will be assumed for pdflatex
%\caption{Simulation Results}
%\label{fig_sim}
%\end{figure}


% An example of a double column floating figure using two subfigures.
% (The subfigure.sty package must be loaded for this to work.)
% The subfigure \label commands are set within each subfigure command, the
% \label for the overall fgure must come after \caption.
% \hfil must be used as a separator to get equal spacing
%
%\begin{figure*}
%\centerline{\subfigure[Case I]{\includegraphics[width=2.5in]{subfigcase1}
% where an .eps filename suffix will be assumed under latex, 
% and a .pdf suffix will be assumed for pdflatex
%\label{fig_first_case}}
%\hfil
%\subfigure[Case II]{\includegraphics[width=2.5in]{subfigcase2}
% where an .eps filename suffix will be assumed under latex, 
% and a .pdf suffix will be assumed for pdflatex
%\label{fig_second_case}}}
%\caption{Simulation results}
%\label{fig_sim}
%\end{figure*}



% An example of a floating table. Note that, for IEEE style tables, the 
% \caption command should come BEFORE the table. Table text will default to
% \footnotesize as IEEE normally uses this smaller font for tables.
% The \label must come after \caption as always.
%
%\begin{table}
%% increase table row spacing, adjust to taste
%\renewcommand{\arraystretch}{1.3}
%\caption{An Example of a Table}
%\label{table_example}
%\centering
%% Some packages, such as MDW tools, offer better commands for making tables
%% than the plain LaTeX2e tabular which is used here.
%\begin{tabular}{|c||c|}
%\hline
%One & Two\\
%\hline
%Three & Four\\
%\hline
%\end{tabular}
%\end{table}


\section{Conclusion}
The conclusion goes here.

% if have a single appendix:
%\appendix[Proof of the Zonklar Equations]
% or
%\appendix  % for no appendix heading
% do not use \section anymore after \appendix, only \section*
% is possibly needed

% use appendices with more than one appendix
% then use \section to start each appendix
% you must declare a \section before using any
% \subsection or using \label (\appendices by itself
% starts a section numbered zero.)
%
% Use this command to get the appendices' numbers in "A", "B" instead of the
% default capitalized Roman numerals ("I", "II", etc.).
% However, the capital letter form may result in awkward subsection numbers
% (such as "A-A"). Capitalized Roman numerals are the default.
%\useRomanappendicesfalse
%
\appendices
\section{Demo software usage}
The coded software is accessible with source codes at http://code.google.com/p/rjimage/. For the compilation and correct work it needs at least Java 1.5 and Ant 1.7 installed at system. So far it correctly works with grayscale (byte for pixel) images and demonstrates the ICM algorithm, Gibbs sampler and Metropolis-Hasting sampler. The RJMCMC methos as I have mentioned is almost done, I am stuck figuring out those $u$'s used in acceptance probability, once it would be over the joy of running RJMCMC algorithm would be provided to community.

% you can choose not to have a title for an appendix
% if you want by leaving the argument blank
%\section{}
%Appendix two text goes here.

% use section* for acknowledgement
%\section*{Acknowledgment}
% optional entry into table of contents (if used)
%\addcontentsline{toc}{section}{Acknowledgment}
%The authors would like to thank...

% trigger a \newpage just before the given reference
% number - used to balance the columns on the last page
% adjust value as needed - may need to be readjusted if
% the document is modified later
%\IEEEtriggeratref{8}
% The "triggered" command can be changed if desired:
%\IEEEtriggercmd{\enlargethispage{-5in}}

% references section
% NOTE: BibTeX documentation can be easily obtained at:
% http://www.ctan.org/tex-archive/biblio/bibtex/contrib/doc/

% can use a bibliography generated by BibTeX as a .bbl file
% standard IEEE bibliography style from:
% http://www.ctan.org/tex-archive/macros/latex/contrib/supported/IEEEtran/bibtex
%\bibliographystyle{IEEEtran.bst}
% argument is your BibTeX string definitions and bibliography database(s)
%\bibliography{IEEEabrv,../bib/paper}
%
% <OR> manually copy in the resultant .bbl file
% set second argument of \begin to the number of references
% (used to reserve space for the reference number labels box)
\begin{thebibliography}{1}

\bibitem{Kato99}
Z.~Kato \emph{Bayesian color image segmentation using reversible jump Markov chain Monte Carlo}.\hskip 1em plus
  0.5em minus 0.4em\relax Probability, Networks and Algorithms, February, 28, 1999. CWI, Amsterdam.

\bibitem{BesagICM}
J.~Besag \emph{On the statistical analysis of dirty pictures}.\hskip 1em plus
  0.5em minus 0.4em\relax Jl.~Roy.~Statist.~Soc.,~ser.~B,~1986.

\bibitem{Metropolis:SA}
N.~Metropolis, A.~Rosenbluth, M.~Rosenbluth, A.~Teller, and E.~Teller \emph{Equation of state calculation by fast computing machines}.\hskip 1em plus
  0.5em minus 0.4em\relax J.~of Chem.~Physics, Vol.~21, pp 1087-1092, 1953.

\bibitem{Hasting}
W.~K.~Hasting \emph{Monte carlo sampling methods using Markov chains and their application}.\hskip 1em plus 0.5em minus 0.4em\relax Biometrika, vol.~57, pp.~1701-1762,~1994.

\bibitem{Green}
P.~J.~Green \emph{Reversible jump Markov chain Monte Carlo computation and Bayesian model determination}.\hskip 1em plus 0.5em minus 0.4em\relax Biometrika, vol.~82, no.~4, pp.~711-731,~1995.

\bibitem{Richardson}
S.~Richardson and P.~J.~Green, \emph{Bayesian analysis of mixtures with an unknown number of components},\hskip 1em plus 0.5em minus 0.4em\relax Jl. Roy. Statist. Soc., ser. B, vol. 59, no. 4, pp. 731-792, 1997.

\bibitem{Barker}
S.~A.~Barker and P.~J.~W.~Rayner, \emph{Unsupervised image segmentation using Markov
random feld models}, \hskip 1em plus 0.5em minus 0.4em\relax Energy Minimization Methods in Computer Vision and Pattern Recognition. 1997, pp. 165-178, Springer.

\bibitem{Geman}
S.~Geman and D.~Geman, \emph{Stochastic relaxation, Gibbs distributions and the Bayesian restoration of images.}\hskip 1em plus 0.5em minus 0.4em\relax  IEEE Trans. on Pattern Analysis and Machine Intelligence, vol. 6, pp. 721-741, 1984.

\end{thebibliography}

% biography section
% 
% If you have an EPS/PDF photo (graphicx package needed) extra braces are
% needed around the contents of the optional argument to biography to prevent
% the LaTeX parser from getting confused when it sees the complicated
% \includegraphics command within an optional argument. (You could create
% your own custom macro containing the \includegraphics command to make things
% simpler here.)
%\begin{biography}[{\includegraphics[width=1in,height=1.25in,clip,keepaspectratio]{mshell}}]{Michael Shell}
% where an .eps filename suffix will be assumed under latex, and a .pdf suffix
% will be assumed for pdflatex; or if you just want to reserve a space for
% a photo:

%\begin{biography}{Michael Shell}
%Biography text here.
%\end{biography}

% if you will not have a photo at all:
%\begin{biographynophoto}{John Doe}
%Biography text here.
%\end{biographynophoto}

% insert where needed to balance the two columns on the last page
%\newpage

%\begin{biographynophoto}{Jane Doe}
%Biography text here.
%\end{biographynophoto}

% You can push biographies down or up by placing
% a \vfill before or after them. The appropriate
% use of \vfill depends on what kind of text is
% on the last page and whether or not the columns
% are being equalized.

%\vfill

% Can be used to pull up biographies so that the bottom of the last one
% is flush with the other column.
%\enlargethispage{-5in}

% that's all folks
\end{document}