% Preamble
\documentclass{article}

\title{Grayscale image segmentation using reversible jump MCMC.}
\author{Pavel Senin}
\date{\today}

% Document
\begin{document}
% create title page and toc
\maketitle
\clearpage
\tableofcontents
1. Introduction.
\clearpage

%%%%%%%%%%%%%%%%%%%%%%%%%%%%%%
% start text here!!
\section{Introduction}

Let's suppose that the observed image is:
$\textsl{F} = \left\{ \vec{f}_{s} \left| s \in \textsl{S}, \forall i : 0 < \vec{f}^{i}_{s} < 1 \right. \right\}$,
where vector $\vec{f}_{s}$ is vector that carried intensity of colour for pixel s. The segmentation itself is just labeling of each pixel $s \in S$ by label $\omega_{s} \in \Lambda = \left\{ 1,2,...,L \right\}$. $\omega\in\Omega$ denotes a labeling (or segmentation), $\Omega$ is a set of all possible labeling.

We regard our image as a sample drawn from unknown Gaussian mixture distribution. The goal of our analysis is inference about the number 
$\textsl{L}$ 
components, the componen parameter 
$\Theta = \left\{\forall\lambda : 1\leq \lambda \leq \textsl{L}, \Theta_{\lambda}=\left(\vec{\mu_{\lambda}},\Sigma_{\lambda} \right) \right\}$ 
the component weights 
$\textsl p_{\lambda}\left(1\leq\lambda\leq \textsl L \right)$
summing to 1, the clique potential (or inter-pixel interaction strength),
$\beta$ 
and the segmentation $\omega$.
\[
so
\]


\end{document}
